
%==============================================================================
% Voorbeeld gebruik documentklasse hogent-article
%==============================================================================
%
% Compileren in TeXstudio:
%
% - Zorg dat Biber de bibliografie compileert (en niet Biblatex)
%   Options > Configure > Build > Default Bibliography Tool: "txs:///biber"
% - F5 om te compileren en het resultaat te bekijken.
% - Als de bibliografie niet zichtbaar is, probeer dan F5 - F8 - F5
%   Met F8 compileer je de bibliografie apart.
%
% Als je JabRef gebruikt voor het bijhouden van de bibliografie, zorg dan
% dat je in ``biblatex''-modus opslaat: File > Switch to BibLaTeX mode.

\documentclass{hogent-article}


%------------------------------------------------------------------------------
% Metadata over het artikel
%------------------------------------------------------------------------------

%---------- Titel & auteur ----------------------------------------------------

% TODO: geef werktitel van je eigen voorstel op
\PaperTitle{Titel van het artikel}
% TODO: geef op welk soort artikel dit is
% Dit is typisch de opdracht en het vak waarvoor dit artikel geschreven is, bv.
% ``Verslag onderzoeksproject Onderzoekstechnieken 2018-2019''
\PaperType{Verslag onderzoeksproject Onderzoekstechnieken 2018-2019}

% TODO: vul je eigen naam in als auteur, geef ook je emailadres mee!
\Authors{Olivier Troch\textsuperscript{1}, Daan Van Vooren \textsuperscript{2}, Robbie Verdurme\textsuperscript{3}, Sebastien Wojtyla\textsuperscript{4}} % Authors



% TODO: vul de naam van je co-promotor in.
% Als het hier gaat om een voorstel voor de bachelorproef, dan ben je hier
% verplicht de naam van je co-promotor in te vullen. Zoniet, dan kan je het
% leeg laten.
\CoPromotor{}

% Contactinfo: Geef hier de contactgegevens van elke auteur van het artikel (en
% indien van toepassing ook van de co-promotor).
\affiliation{
	\textsuperscript{1} \href{mailto:Olivier.troch.w2257@student.hogent.be}{Olivier.troch.w2257@student.hogent.be}
}
\affiliation{
	\textsuperscript{2} \href{mailto:daan.vanvooren.y1502@student.hogent.be}{daan.vanvooren.y1502@student.hogent.be}
}
\affiliation{
	\textsuperscript{3}
	\href{mailto:robbie.verdurme.y9234@student.hogent.be}{robbie.verdurme.y9234@student.hogent.be}
}
\affiliation{
	\textsuperscript{4}
	\href{mailto;sebastien.wojtyla.y3274@student.hogent.be}{sebastien.wojtyla.y3274@student.hogent.be}
}
%---------- Abstract ----------------------------------------------------------

\Abstract{In deze paper wordt onderzocht wat de effecten zijn op de resultaten van de retrieval practice studiemethode.
	In veel studies werd reeds aangetoond dat retrieval practice een goede studiemethode is maar welke factoren
	hierop invloed hebben is minder besproken. De paper gaat dieper in op deze vraag door te onderzoeken of het
	krijgen van /*** VARIABLEN INVULLEN ***/ gedurende de retrieval practice methode enige invloed zal hebben op de
	resultaten. De te verwachte resultaten zijn dat zowel de /*** VARIABLEN INVULLEN ***/ een positief effect zullen hebben
	en dus betere testresultaten zullen opleveren. Deze paper kan bijdragen aan verder onderzoek van de retrieval
	practice methode.
}

%---------- Onderzoeksdomein en sleutelwoorden --------------------------------
% TODO: Vul de sleutelwoorden aan.


\Keywords{Onderzoeksproces, Studiemethodes, Retrieval Practice}
\newcommand{\keywordname}{Sleutelwoorden} % Defines the keywords heading name

%---------- Titel, inhoud -----------------------------------------------------

\begin{document}
	
	\flushbottom % Makes all text pages the same height
	\maketitle % Print the title and abstract box
	\tableofcontents % Print the contents section
	\thispagestyle{empty} % Removes page numbering from the first page
	
	%------------------------------------------------------------------------------
	% Hoofdtekst
	%------------------------------------------------------------------------------
	
	\section{Inleiding}
	
	
	
	\section{Overzicht literatuur}
	
	% Refereren naar de literatuur kan met:
	% \autocite{BIBTEXKEY} -> (Auteur, jaartal)
	% \textcite{BIBTEXKEY} -> Auteur (jaartal)
	Voorbeeld van een referentie~\autocite{Moore2002}
	
	
	
	\section{Methodologie}
	

	
	\section{Experimenten}
	Tijdens het onderzoek krijgt iedere informatica student een tekst die hij/zij zal moeten bestuderen. Vervolgens wordt de groep opgesplitst in drie subgroepen. Elke subgroep zal de standaard retrieval practice methode (STST) toepassen waarvan twee subgroepen met een aangepaste variabele.
	De eerste subgroep zal de standaard retrieval practice methode toepassen zonder aangepaste variabele. 
	
	%Variable 1 bespreken
	De tweede subgroep zal tussen verschillende studie iteraties verplaatst worden naar een andere /*** VARIABLE 1 ***/ waar ook de standaard retrieval practice zal toegepast worden.
	
	%Variable 2 bespreken 
	De laatste subgroep zal tijdens het studeren /*** VARIABLE 2 ***/ krijgen op de resultaten van hun voorgaande test.
	
	Hierdoor kan nagegaan worden of de variabelen een invloed hebben op het studeren van een tekst aan de hand van de retrieval practice methode. Merk op dat we elke tweedejaars informatica student er toe verplichten om deel te nemen aan deze test waardoor het resultaat niet op de hele groep studenten toepasbaar is zonder enig foutpercentage \autocite{karpicke2009metacognitive}.
	
	We veronderstellen dat de eerste subgroep analoge resultaten zal behalen aan de resultaten uit de artikels over de retrieval practice methode \autocite{butler2010repeated, pyc2012test, karpicke2007repeated, karpicke2008critical}. Dit is omdat deze methode reeds vaak getest werd in verschillende experimenten.
	
	%verwachtingen Variable bespreken
	Daarnaast verwachten we ook dat zowel het geven van /*** VARIABLE 2 ***/ als aanpassen van de /*** VARIABLE 1 ***/ een positief effect zal hebben op het resultaat en waarbij het geven van /*** VARIABLE 2 ***/ het grootste effect zal hebben.
	
	
	\section{Analyse resultaten}
	
	
	\section{Conclusie}
	
	
	
	%------------------------------------------------------------------------------
	% Referentielijst
	%------------------------------------------------------------------------------
	% TODO: de gerefereerde werken moeten in BibTeX-bestand ``bibliografie.bib''
	% voorkomen. Gebruik JabRef om je bibliografie bij te houden en vergeet niet
	% om compatibiliteit met Biber/BibLaTeX aan te zetten (File > Switch to
	% BibLaTeX mode)
	
	\phantomsection
	\printbibliography[heading=bibintoc]
	
\end{document}
