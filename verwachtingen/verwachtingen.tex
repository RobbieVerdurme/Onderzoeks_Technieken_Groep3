% !TeX spellcheck = en_GB
%==============================================================================
% Voorbeeld gebruik documentklasse hogent-article
%==============================================================================
%
% Compileren in TeXstudio:
%
% - Zorg dat Biber de bibliografie compileert (en niet Biblatex)
%   Options > Configure > Build > Default Bibliography Tool: "txs:///biber"
% - F5 om te compileren en het resultaat te bekijken.
% - Als de bibliografie niet zichtbaar is, probeer dan F5 - F8 - F5
%   Met F8 compileer je de bibliografie apart.
%
% Als je JabRef gebruikt voor het bijhouden van de bibliografie, zorg dan
% dat je in ``biblatex''-modus opslaat: File > Switch to BibLaTeX mode.

\documentclass{hogent-article}
%------------------------------------------------------------------------------
% Metadata over het artikel
%------------------------------------------------------------------------------

%---------- Titel & auteur ----------------------------------------------------

% TODO: geef werktitel van je eigen voorstel op
\PaperTitle{Hebben de factoren feedback en locatie invloed op de resultaten van de retrieval practice studiemethode?}
% TODO: geef op welk soort artikel dit is
% Dit is typisch de opdracht en het vak waarvoor dit artikel geschreven is, bv.
\PaperType{Verslag onderzoeksproject Onderzoekstechnieken 2018-2019}

% TODO: vul je eigen naam in als auteur, geef ook je emailadres mee!
\Authors{Olivier Troch\textsuperscript{1}, Daan Van Vooren \textsuperscript{2}, Robbie Verdurme\textsuperscript{3}, Sebastien Wojtyla\textsuperscript{4}} % Authors

% TODO: vul de naam van je co-promotor in.
% Als het hier gaat om een voorstel voor de bachelorproef, dan ben je hier
% verplicht de naam van je co-promotor in te vullen. Zoniet, dan kan je het
% leeg laten.
\CoPromotor{}

% Contactinfo: Geef hier de contactgegevens van elke auteur van het artikel
\affiliation{
	\textsuperscript{1} \href{mailto:Olivier.troch.w2257@student.hogent.be}{Olivier.troch.w2257@student.hogent.be}
}
\affiliation{
	\textsuperscript{2} \href{mailto:daan.vanvooren.y1502@student.hogent.be}{daan.vanvooren.y1502@student.hogent.be}
}
\affiliation{
	\textsuperscript{3}
	\href{mailto:robbie.verdurme.y9234@student.hogent.be}{robbie.verdurme.y9234@student.hogent.be}
}
\affiliation{
	\textsuperscript{4}
	\href{mailto;sebastien.wojtyla.y3274@student.hogent.be}{sebastien.wojtyla.y3274@student.hogent.be}
}

%---------- Abstract/Samenvatting ----------------------------------------------------------
\Abstract{Hier schrijf je de samenvatting van je artikel, als een doorlopende tekst van één paragraaf. Wat hier zeker in moet vermeld worden: \textbf{Context} (Waarom is dit werk belangrijk?); \textbf{Nood} (Waarom moet dit onderzocht worden?); \textbf{Taak} (Wat ga je (ongeveer) doen?); \textbf{Object} (Wat staat in dit document geschreven?); \textbf{Resultaat} (Wat verwacht je van je onderzoek?); \textbf{Conclusie} (Wat verwacht je van van de conclusies?); \textbf{Perspectief} (Wat zegt de toekomst voor dit werk?).
	
	Bij de sleutelwoorden geef je het onderzoeksdomein, samen met andere sleutelwoorden die je werk beschrijven.
}

%---------- Onderzoeksdomein en sleutelwoorden --------------------------------
% TODO: Vul de sleutelwoorden aan.
\Keywords{Onderzoeksproces, Leerstrategieën, Retrieval Practice}
\newcommand{\keywordname}{Sleutelwoorden} % Defines the keywords heading name

%---------- Titel, inhoud -----------------------------------------------------
\begin{document}
	
	\flushbottom % Makes all text pages the same height
	\maketitle % Print the title and abstract box
	\tableofcontents % Print the contents section
	\thispagestyle{empty} % Removes page numbering from the first page
	
	%------------------------------------------------------------------------------
	% Hoofdtekst
	%------------------------------------------------------------------------------
	
	\section{Inleiding}
	Retrieval practice is een studiemethode die ervoor zorgt dat leerstof langer onthouden kan worden op lange termijn. Hoewel reeds aangetoond is dat dit een effectieve methode is voor het studeren verwachten wij een ander resultaat wanneer we een paar variabelen aanpassen. Eén van de variabelen die we zullen testen in deze paper is de locatie waar gestudeerd wordt en de invloed hiervan op het studeren. 
	Een andere variabele die mogelijks ook invloed heeft op de efficiëntie van deze studiemethode is het geven van feedback tussen de verschillende test iteraties.
	
	\section{Literatuurstudie}
	Er zijn veel verschillende methoden om te studeren. Een van die methoden is de retrieval practice. Deze focust op het verschil tussen het testen van de geleerde leerstof en het leren ervan.
	Tulving \autocite{} heeft deze al uitbunding bestudeerd. Er zijn 4 verschillende methoden die Tulving getest heeft. De studie-test-studie-test methode. Deze laat de persoon eerst de stof bestuderen om erna zichzelf te testen, dit gebeurd 2 keer voor dat de finale test word afgenomen een week later. Er zijn ook nog de 3 andere methoden deze zijn (SSSS, SSST, STTT).
	Tussen deze studiemethoden werd er al een groot verschil gevonden op het langetermijngeheugen. Namelijk dat de standaard methode (STST) de persoon de stof na een week nog grotendeels kan reconstrueren. Bij de andere methoden zagen we duidelijk een verschil tussen studeren en testen. Hoe meer testen de persoon had afgelegd hoe beter deze over het algemeen scoorde tegenover de andere personen die maar enkel 1 test hadden. Hiervan kunnen we afleiden dat er een groot verschill ligt tussen het studeren en het testen of bevragen van de stof.
	Wij willen weten aan de hand van een paar variablen of deze stelling nog altijd klopt. Hiervoor hebben we een aantal artikels opgezocht die te maken hebben met het studeren op verplaatsing. Dit hebben we eveneens gedaan voor onze andere variable namelijk of de persoon feedback krijgt achter het testen.
    
    \title{Test-Enhanced Learning: The Potential for Testing to Promote Greater Learning in Undergraduate Science Courses}
    Deze test basseert zich op de studie van \autocite{} maar verdiept zich ook op enkele variabelen die de uitkomst beïnvloeden. Eén van deze varaiabelen is de importantie van het behalen van een goed resultaat. Hier bedoeld de studie mee dat een test die een grote invloed heeft veel belangrijker is en dus ook de drijfsweer van de tester beïnvloedt.
    Ze testen ook op het geven van feedback. Het geven van feedback geeft aan wat de tester juist of fout had en versterkt zo de materiekennis en zelfkennis.
	% Refereren naar de literatuur kan met:
	% \autocite{BIBTEXKEY} -> (Auteur, jaartal)
	% \textcite{BIBTEXKEY} -> Auteur (jaartal)
	Voorbeeld van een referentie~\autocite{karpicke2008critical}
	
	\section{Variablen}
	De variablen dat worden getest in deze paper zijn de volgende.
	De eerste variable die invloed zou kunnen hebben is de locatie waarop je studeert en de test uitvoert.
	[olivier waarom is dit zo?]
	
	De tweede variable die invloed zou kunnen hebben is het geven van feedback tussen de verschillende test iteraties. 
	Dit is zodat de test persoon dan weet welke fouten hij/haar gemaakt heeft. Hieruit kan de test persoon rekening houden met de volgende studie iteratie. Zodat deze fout niet meer kan voorkomen. 
	
	\section{Mockgrafieken}
	
	\section{Conclusie}
	
	\section{Bibliografie}
	%------------------------------------------------------------------------------
	% Referentielijst
	%------------------------------------------------------------------------------
	% TODO: de gerefereerde werken moeten in BibTeX-bestand ``bibliografie.bib''
	% voorkomen. Gebruik JabRef om je bibliografie bij te houden en vergeet niet
	% om compatibiliteit met Biber/BibLaTeX aan te zetten (File > Switch to
	% BibLaTeX mode)
	
	\phantomsection
	\printbibliography[heading=bibintoc]
	
\end{document}
